\documentclass{article}
\author{Chaoyu Gao/cgao15@ucsc.edu \\ Gongyu Sun/gosun@ucsc.edu}
\title{ HW5 Applying Hoare Logic }
\usepackage{amsmath}
\begin{document}
\maketitle
\leftline {\textbf{CMPS 203}}
\leftline {On this homework, we worked together for 4 hours.}
\section{  }

Using Hoare's rules, prove:
	\[ \{x = y\} x := x + 1; y := y + 1 \{x = y\} \]

	\textbf {Proof:}
	\begin{align*}
	WP(c_1, c_2, Q) &= (x := x + 1, WP(y := y+1, x = y)) \\
				  &= (x := x + 1, x = y + 1) \\
				  &= (x + 1 = y + 1) \\
				  &= (x = y)
	\end{align*}
	\begin{center}
	We have P = \{x = y\}  = WP($c_1$, $c_2$, Q)\\
	Hence we have:
	 \centerline {$\{x = y\}$ x := x + 1; y := y + 1 $\{x = y\}$ }
	 Or use the following structure:\\
	 $$\frac 
	 	{\frac
			{\frac
			{}{\displaystyle \{ x + 1 = y + 1\}x:= x + 1\{x = y + 1\}}Ass
			\frac
			{}{\displaystyle \{x = y + 1\}y:= y+ 1\{x = y\} }Ass
			}
			{\displaystyle \{x = y\}\Rightarrow \{x + 1 = y + 1\},\ \{x + 1 = y + 1\}\ x := x + 1;\ y := y + 1\{x = y\} }
			Seq
		}
		 {\{x = y\} x := x + 1; y := y + 1 \{x = y\}}
	 	Conseq$$
	Therefore we have:\\
	\centerline {$\{x = y\}$ x := x + 1; y := y + 1 $\{x = y\}$ }
	\end{center}
	
\section{}
	Using Hoare's rules, prove:
	 	\[ \{y = z\}\ while\  b\  do\  y := y - x\ \{ \exists k.z = (y + k \ast x)\} \]
	 \textbf {Proof:}
	 \begin{center}
	Let loop = \{while b do y := y - x\}\\
	We can see that the invariant is\\ I = $\{ \exists k.z = (y + k \ast x)\}$ \\
	Also, let J = $\{ \exists k.z = ((y - x)+ k \ast x)\}$
	$$\frac
		{\frac
			{\frac
				{\frac
					{}
					{\displaystyle \{ b \land I\} \Rightarrow  \{ b \land J\},\{ b \land J\}\ y := y -x\ \{I\}}\displaystyle Ass
				}
				{\displaystyle \{b \land  \exists k.z = (y + k \ast x)\} \ y := y -x\ \{ \exists k.z = (y + k \ast x)\} }\displaystyle Conseq
			}
			{\displaystyle \ \{ I \}\ loop\ \{ \neg b \land I\}}While\ \ \ \ \   \{ \neg b \land I\} \Rightarrow \{I\}
		}
		{\displaystyle\{y = z\}\ while\  b\  do\  y := y - x\ \{ \exists k.z = (y + k \ast x)\}}
		Conseq$$
	Hence we have:
	\[ \{y = z\}\ while\  b\  do\  y := y - x\ \{ \exists k.z = (y + k \ast x)\} \]
	\end{center}
	
\end{document}
